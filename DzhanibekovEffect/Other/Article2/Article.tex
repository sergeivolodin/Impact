\documentclass[14pt,a4paper]{extarticle}
       \usepackage{mathtext}
       \usepackage{amsmath,amssymb}
       %\usepackage[T2A]{mathtext}
       \usepackage[T2A]{fontenc}
       \usepackage[utf8]{inputenc}
       \usepackage[russian]{babel}
       %\usepackage[math]{pscyr}
       \usepackage{wrapfig}
       %\usepackage[dviwin]{graphicx}
       \usepackage[dvips]{graphicx}
       \usepackage{hyperref}
       \textwidth=160mm
       \textheight=230mm
       %\oddsidemargin=0mm
       \topmargin=-10mm
       \sloppy \large
       \sloppy

%\numberwithin{equation}{section}
%\newcommand{\sectio}[1]{
%\par
%\refstepcouner{section}
%{\normalsize\rmfamily\bfseries\upshape
%\thesection. #1.}
%}
\sloppy

\def\p{\partial}
\def\DS{\displaystyle}
\def\eps{\varepsilon}

\newcounter{FORMULA}
\renewcommand{\theFORMULA}{\arabic{FORMULA}}
       
       \def\Form#1$$#2$${\refstepcounter{FORMULA}%    %% здесь есть
%\\{\tt{#1}}                                  %% грязный трюк
$$#2\eqno(\theFORMULA)\label#1$$}              %% с label


\large
\renewcommand{\baselinestretch}{1.5}
\begin{document}

УДК 531.37

\smallskip


\centerline{2013~г.~{А.~Г.~Петров, С.~Е.~Володин}}

\smallskip

\centerline{\bf <<ЭФФЕКТ ДЖАНИБЕКОВА>> И ЗАКОНЫ МЕХАНИКИ}

\bigskip

ВВЕДЕНИЕ

Космонавт В.А.Джанибеков в 1985 г. в условиях невесомости наблюдал, как закрученная и отпущенная в пространство гайка начинала прямолинейное движение с вращением вокруг своей оси. Внезапно без видимых причин её ось поворачивалась на $ 180^\circ$. Эти <<кувырки>> повторялись через равные промежутки времени. Выглядело эффектно и загадочно \cite{video}.  На некоторых сайтах в Интернете (\cite{federalspace}, \cite{ixbt})  появились довольно необоснованные объяснения этого явления, а также были высказаны опасения относительно его возможности для нашей планеты. Например, на сайте Роскосмоса утверждается: «открытие Джанибекова послужило толчком к развитию квантовых исследований макромира», «появилась даже гипотеза, что точно так на своем очередном витке по орбите может совершить кувырок и наша планета». В действительности же ничего нового в таком движении твёрдого тела нет. Это простейший случай движения твёрдого тела около неподвижной точки при отсутствии внешнего момента силы --- случай Эйлера. Полное решение задачи описывается эллиптическими функциями, что, видимо, и является препятствием к ее пониманию. Однако, для описания частного случая движения, который наблюдал В. А. Джанибеков достаточно  знаний в пределах начального курса механики, которые даются студентам на первом или втором курсе любого технического ВУЗа.

Вектор кинетического момента неподвижен в пространстве, так как можно считать, что на тело не действуют внешние силы. Но в подвижных осях твёрдого тела этот вектор описывает траекторию, которая является пересечением некоторых сферы и эллипсоида с общим центром. В опыте В. А. Джанибекова такая траектория близка к полуокружностям, на которых между осью гайки и неподвижным вектором кинетического момента (угол нутации) периодически меняется от нуля до $180^\circ$ и обратно. Зависимость угла нутации в предельном случае выражается через элементарную функцию времени.

КИНЕМАТИЧЕСКИЕ И ДИНАМИЧЕСКИЕ ХАРАКТЕРИСТИКИ ТВЕРДОГО ТЕЛА

В опыте Джанибекова на твёрдое тело не действуют внешние силы. Поэтому его центр масс $C$ движется прямолинейно и равномерно. Поместим начало инерциальной системы отсчёта в центр масс и будем изучать в ней движение твёрдого тела. Точка $C$ твёрдого тела является неподвижной  точкой в этой системе отсчёта.

Для описания движения твёрдого тела около неподвижной точки $C$ кроме инерциальной системы отсчёта вводится подвижная, скреплённая с твёрдым телом система координат $\xi,\eta,\zeta$ с единичными базисными векторами ${\bf e}_1,{\bf e}_2,{\bf e}_3$. Ось $\eta$ и соответствующий ей базисный вектор ${\bf e}_2$ направим по оси гайки. Оси $\xi,\eta,\zeta$ являются главными осями тензора инерции твёрдого тела (рис. 1а)

Приведём общие формулы для  вектора угловой скорости ${\omega}$, кинетического момента ${\bf K}$ и кинетической энергии $T$
\Form{general}
$$
\begin{array}{l}
\omega=p{\bf e}_1+q{\bf e}_2+r{\bf e}_3,\quad {\bf K}=X{\bf e}_1+Y{\bf e}_2+Z{\bf e}_3,\quad
2T = X p+Y q+Z r
\end{array}
$$
где $X=Ap,\,Y=Bq,\,Z=Cr$ -- компоненты ${\bf K}$ и $A,B,C$ -- моменты инерции твёрдого тела, причем $A>B>C$.

В опыте Джанибекова интересно найти закон изменения угла нутации $\theta$ между неподвижным вектором ${\bf K}$ и осью ${\bf e}_2$ (рис. 2 а). Проектируя вектор ${\bf e}_2$ на вектор ${\bf K}$, получим
\Form{general1}
$$
\cos\theta=Y/K,\quad K=\sqrt{X^2+Y^2+Z^2}
$$
ЗАКОНЫ СОХРАНЕНИЯ 

При отсутствии внешних сил вектор кинетического момента ${\bf K}$ неподвижен в пространстве. Оси твёрдого тела ${\bf e}_1,{\bf e}_2,{\bf e}_3$ --- подвижны. Поэтому компоненты вектора ${\bf K}$ в подвижных осях будут меняются со временем, однако квадрат длины вектора сохраняется. Коме того, будет сохраняться кинетическая энергия твердого тела $T$. Эти два закона сохранения имеют вид:
\Form{cons1}
$$
X^2+Y^2+Z^2=K^2,\quad  X^2/A+Y^2/B+Z^2/C=2T
$$
Таким образом, траектория, которую описывает вектор ${\bf K}$ в подвижных осях, является пересечением сферы и эллипсоида.
Такое представление траекторий называется геометрической интерпретацией Мак-Кулага \cite{jur}.

На рис. 2 изображено семейство траекторий вектора ${\bf K}$ на эллипсоиде Мак-Кулага. 
На рисунке видно, что если в начальный момент вектор ${\bf K}$ находится на оси $\eta$ со средним значением момента инерции, то его траекториями являются сепаратрисы (жирные линии). Движение происходит от полюса $B$ до противоположного полюса $B'$. Угол нутации $\theta$ меняется от $0$ до $180^\circ$. Это движение и объясняет <<кувырок>> в эксперименте В. А. Джанибекова. 

Если же в начальный момент совместить  вектор ${\bf K}$ с осью $\xi$ или $\zeta$, соответствующие большой и малой осям инерции, то вектор ${\bf K}$ смещаться не будет. При малом отклонении вектора ${\bf K}$ от осей $\xi$ или $\zeta$ он опишет замкнутую траекторию малого радиуса. Этот эффект тоже демонстрировал В.А. Джанибеков. Он стукал молотком по оси вращающегося волчка и ось практически не откланялась. Наша Земля вращается около оси с наибольшим значением момента инерции и поэтому опасности <<кувырка>> для нее нет. 

Однако, найти зависимость от времени угла нутации с помощью только законов сохранения (\ref{cons1}) нельзя. Для этого нужно привлечь уравнение изменения одной из компонент вектора ${\bf K}$.  

ДВИЖЕНИЕ ПО СЕПАРАТРИСАМ

Пусть в начальный момент угловая скорость направлена по оси ${\bf e}_2$ и $Y=K,\;Y^2/B=2T$, откуда следует $K^2=2TB$. Умножая второе уравнение (\ref{cons1}) на $B$ и вычитая из первого, получим уравнение двух плоскостей
$$ 
(1-B/A)X^2-(B/C-1)Z^2=0\Rightarrow X\sqrt{1-B/A}\pm Z\sqrt{B/C-1}=0.
$$ 
Таким образом, при выполнении условия $K^2=2TB$ траектории вектора ${\bf K}$ являются пересечением плоскостей и эллипсоида Мак-Кулага. Это четыре полуокружности между полюсами эллипсоида $B$ и $B'$. Они называются сепаратрисами и изображены жирными линиями  на рис. 2. а)~--- на эллипсоиде Мак-Кулага, б)~--- в развертке.

Спроектировав закон сохранения кинетического момента в подвижных осях $d{\bf K}/dt+\omega\times K=0$ на ось ${\bf e}_2$, получим 
\Form{cons3}
$$
\frac{dY}{dt}=-(A-C)\,\frac{XZ}{AC}
$$   
Подставляя вместо $X$ и $Z$ выражения компонент  вектора кинетического момента $X$ и $Z$, найденные
из законов сохранения (\ref{cons1}) 
\Form{cons2}
$$
X^2=\frac{B-C}{A-C}\,\frac{A}{B}(K^2-Y^2),\quad
Z^2=\frac{A-B}{A-C}\,\frac{C}{B}(K^2-Y^2),
$$
получим уравнение для $Y$
$$
\frac{dY}{dt}=\pm\frac{\alpha}{K}\left(K^2-Y^2\right), \quad \alpha=\frac{K}{B}\,\sqrt{\frac{(A-B)(B-C)}{AC}}
$$
Решение уравнения для $Y$ выражается через гиперболический тангенс, а $X$ и $Y$ находятся из (\ref{cons2}) \cite{appel}
$$
\begin{array}{c}
\DS X=\pm K\,\sqrt{\frac{(B-C)A}{B(A-C)}}\,\frac{1}{\ch\alpha t},\quad
\DS Z=\pm K\,\sqrt{\frac{(A-B)C}{B(A-C)}}\,\frac{1}{\ch\alpha t},\\
\DS Y=\pm K\th\alpha t
\end{array}
$$

Согласно (\ref{general1}) угол нутации $\theta$ меняется в интервале $(0, \pi)$ по закону
\Form{theta}
$$
\cos\theta=\pm\th\alpha t,
$$
поэтому движение вектора ${\bf K}$ от полюса $B'$ к полюсу $B$ происходит по одной из четырех полуокружностей за бесконечно большое время.

ИЗМЕНЕНИЕ УГЛА НУТАЦИИ ПРИ МАЛОМ ОТКЛОНЕНИИ ОТ СЕПАРАТРИСЫ.

Если имеется малое отклонение от сепаратрисы, которое можно охарактеризовать параметром
 $
 \eps=K^2-2TB,
 $
то  вектор ${\bf K}$ описывает замкнутую траекторию $MNN'M'$ близкую к сепаратрисам $BB'$  (рис.2) за конечное время. Половина этого времени, то есть, полупериод движения твердого тела $t_0$ находится следующим образом.

Из законов сохранения (\ref{cons1}) выражаем $X$ и $Z$ через $Y$  
$$
\begin{array}{c}
\DS X^2=\frac{B-C}{A-C}\,\frac{A}{B}(K^2-Y^2)+\frac{AC}{B(A-C)}\,\eps,\\[2ex]
\DS Z^2=\frac{A-B}{A-C}\,\frac{C}{B}(K^2-Y^2)-\frac{AC}{B(A-C)}\,\eps
\end{array}
$$
и подставляем их в уравнение (\ref{cons3}). Из него полупериод $t_0$ выражается через интеграл
\Form{t0}
$$\DS t_0=2\frac{AC}{A-C}\int_0^{Y_0}\frac{dY}{X(Y)\,Z(Y)}$$

Верхний предел интеграла $Y_0$ является соответствующим знаку $\varepsilon$ корнем уравнения $X(Y)=0$ либо $Z(Y)=0$. 
Для вычисления полупериода при $\eps\rightarrow 0$ удобна асимптотика (см. приложение)
\Form{alphat0} 
$$
\DS\alpha t_0=\ln{\frac{16}{|\mu|}}+O(\mu\ln|\mu|), \quad \mu=\frac{\eps(A-C)B}{K^2(A-B)(B-C)}+o(\varepsilon)
$$ 
Зависимость $\cos\theta$ от времени можно скомбинировать с помощью функции (\ref{theta}), учитывая найденный полупериод $t_0$, т.к. вдали от полюсов $B$, $B'$ характер движения слабо зависит от $\eps$. На интервале $t\in (-t_0,5t_0)$ эта зависимость следующая (см. рис. 3):
$$
\cos\theta=\th\alpha t - \th\alpha (t-t_0)+\th\alpha( t-2t_0)-\th\alpha (t-3t_0)+\th\alpha (t-4t_0)
$$

Наблюдатель в неподвижном пространстве увидит попеременное изменение направления средней оси твёрдого тела на противоположное. Поскольку скорость конца вектора $K$ вблизи полюсов близка к нулю, то средняя ось достаточно долго  задерживается вблизи полюса (дуга $MN$ на рис. 2). При прохождении дуги $NN'$ совершается кувырок, при котором  $\cos\theta$ меняется от значения, близкого к $1$, до значения, близкого к $-1$. В точках $N$ и $N'$ условимся считать, что $\cos\theta=0.9$ и $\cos\theta=-0.9$ соответственно. Время кувырка $\tau$ на дуге $NN'$, за которое $\cos\theta$ меняется от значения $0.9$ до  $-0.9$, найдётся из решения уравнения $\th(\alpha\tau/2)=0.9$. Это значение приблизительно равно $\tau=3/\alpha$.
Отношение $\tau$ к полупериоду движения при малом $\mu$ составляет величину
$
{\tau}/{t_0}=3\left(\ln({16}/{|\mu|})\right)^{-1}.
$

Пусть в начальный момент твёрдое тело закручено около средней главной оси c угловой скоростью $q_0$ по оси $\eta$ с малыми случайными отклонениями $p_0, r_0$  по двум другим осям  $p_0^2+r_0^2<<q_0^2$. Тогда для модуля кинетического момента и параметров $\eps$ и $\mu$ имеем
$$
\begin{array}{c}
\DS K=Bq_{0},\quad \eps =A\left( A-B\right) p_{0}^{2}-C\left( B-C\right)
r_{0}^{2},\\[2ex]
\DS \mu = \frac{A-C}{B}\left(\frac{A}{B-C}\frac{p_0^2}{q_0^2}-\frac{C}{A-B}\frac{r_0^2}{q_0^2}\right)
\end{array}
$$
По наблюдаемой величине $\tau/t_0$ можно оценить начальное отклонение направления угловой скорости от главной оси.  Если, для примера \cite{video}, отношение времени кувырка к полупериоду равно $\tau/t_0=0.4$, то, используя вычисленные моменты инерции (см. приложение) и полагая $r_0=0$, можно получить $\mu\approx 0.009,\;{p_0}/{q_0}\approx 0.08$.  

\smallskip

ПРИЛОЖЕНИЕ

{\bf Расчёт главных моментов инерции.} Для приближенной оценки главных моментов инерции заменим изучаемую гайку цилиндром с внутренним и внешним радиусами цилиндрами $r_1,\,r_2$, высотой $h$ и массой $m_1$ и двумя симметрично расположенными относительно оси цилиндра точечными массами $m_2$ на расстоянии $a$ от оси цилиндра. Основание перпендикуляра от точечной массы $m_2$ на ось цилиндра находится на расстоянии $b$ от центра цилиндра. Массы $m_2$ крепятся к цилиндру стержнями, массы которых пренебрежимо малы (рис. 1б). 

Центр масс такой системы $C$ находитcя на расстоянии $b_1$ от центра цилиндра $O$
$$
b_1=OC=\frac{2m_2}{2m_2+m_1}\,b,
$$
где масса $m_1=\rho\pi h(r_2^2-r_1^2).$
Центр системы координат $\xi,\eta,\zeta$ поместим в центр масс. Ось $\eta$ направлена по оси цилиндра, а ось $\zeta$ расположена в плоскости  оси цилиндра и точечных масс $m_2$. 

Определяя моменты инерции относительно осей $\xi,\eta,\zeta$ цилиндра: 
$$
A_1=C_1=\frac{m_1}{4}(r_1^2+r_2^2+\frac{h^2}{3}+4b_1^2),\quad B_1=\frac{m_1}{2}(r_1^2+r_2^2)
$$
и двух масс $m_2$: $
A_2=2m_2\left(a^2+(b-b_1)^2\right),\quad B_2=2m_2a^2,\quad C_2=2m_2(b-b_1)^2,
$
находим главные моменты инерции всей системы 
$
A=A_1+A_2,\quad B=B_1+B_2,\quad C=C_1+C_2.
$

Для проведения вычислений возьмём $\rho=7800\, kg/m^3,\;r_1=5\times 10^{-3}m,\;r_2=7\times 10^{-3}m,\;h=5\times 10^{-3}m,\;a=10^{-2}m,\;b=8\times 10^{-3}m$. Найдем $b_1=5\times 10^{-3}m$ и $m_1=3\times 10^{-3}kg$, примем $m_2=m_1$.
Выполнив вычисления, получим
$$A=8\times 10^{-7}kg\,m^2,\;B=7\times 10^{-7}kg\,m^2,\;C=2\times 10^{-7}kg\,m^2$$

{\bf Расчет периода.} Знак $\varepsilon$ определяет вид кривой, описываемой вектором $\vec{K}$: при положительных значениях эта кривая замкнута вокруг оси $\xi$, а при отрицательных~--- вокруг оси $\zeta$. Поэтому при $\varepsilon>0$ время достигает четверти периода при $Z(Y)=0$, а при $\varepsilon<0$~--- при $X(Y)=0$. Рассмотрим первый случай $\eps>0$. Заменой переменной 
$
Y=x\sqrt{K^2-\frac{A}{A-B}\eps}
$
интеграл для полупериода (\ref{t0})  приводится к виду
$$
\DS \alpha t_0=2\sqrt{1+\frac{B-C}{A-C}\mu}\int\limits_0^1\frac{dx}{\sqrt{(1-x^2)(1+\mu-x^2)}}
$$
где $\mu$ выражается через параметр $\eps$:
$$\mu =\frac{\left( A-C\right) B\eps}{\left(
B-C\right) \left( \left( A-B\right) K^{2}-A\eps\right) }.$$

Для интеграла можно получить асимптотическое разложение по малому параметру $\mu\rightarrow +0$
и с помощью него находится асимптотическое разложение для полупериода 
$$\alpha t_{0}=
\left( \ln \frac{16}{\mu }\right) \left[ 1+\left( \frac{1}{2\left( \ln \frac{%
16}{\mu }\right) }-\frac{A-2B+C}{4\left( A-C\right) }\right) \mu +O(\mu ^{2})%
\right] $$
В этом двухчленном разложение главная асимптотика совпадает с (\ref{alphat0}), а второе слагаемое в скобках определяет относительную ошибку.
Аналогично заменой $Y=x\sqrt{K^2+\frac{C}{B-C}\,\varepsilon}$ находится асимптотика (\ref{alphat0}) при $\eps<0$. 

{ВЫВОДЫ} 


\smallskip
Таким образом, движение, сопровождаемое периодическими <<кувырками>>, описывается уравнениями Леонарда Эйлера, полученными и изученными им более 250 лет назад. В условиях невесомости этот эффект проявляется более ярко, на что и обратил внимание космонавт В.А. Джанибеков. Нового эффекта здесь нет, никаких новых законов физики для объяснения движения привлекать не надо. Что же касается возможности <<кувырка>> Земли, то эти опасения беспочвенны. Её вращение происходит около оси с наибольшим моментом инерции и является устойчивым.    


\smallskip

Авторы благодарят В.Ф. Журавлева и Д.М. Климова за внимание к работе.

\begin{thebibliography}{99}
\bibitem{jur} Журавлев~В.~Ф. Основы теоретической механики.
М.:~Наука Физматлит, 1997, 320~с.

\bibitem{appel} Аппель П. Теоретическая механика. Том 2. М.: Физ-мат лит. 1960. 487 с.
\bibitem{video} Демонстрация эффекта на youtube.com:\newline http://youtu.be/60iBwQwAnqo?t=1m30s
\bibitem{federalspace} Описание на сайте Федерального космического агентства:\newline http://www.federalspace.ru/main.php?id=2\&nid=19068
\bibitem{ixbt} Обсуждение явления на ixbt.com:\newline http://forum.ixbt.com/topic.cgi?id=64:1310
\bibitem{numerical} Численный расчет движения, использующий уравнения Эйлера и визуализация результатов:\newline http://youtu.be/7QeCQxGj38Q, http://youtu.be/VHNvzXy-Iqs
\end{thebibliography}
\end{document}

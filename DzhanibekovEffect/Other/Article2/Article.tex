\documentclass[14pt,a4paper]{extarticle}
       \usepackage{mathtext}
       \usepackage{amsmath,amssymb}
       %\usepackage[T2A]{mathtext}
       \usepackage[T2A]{fontenc}
       \usepackage[utf8]{inputenc}
       \usepackage[russian]{babel}
       %\usepackage[math]{pscyr}
       \usepackage{wrapfig}
       %\usepackage[dviwin]{graphicx}
       \usepackage[dvips]{graphicx}
       \textwidth=160mm
       \textheight=230mm
       %\oddsidemargin=0mm
       \topmargin=-10mm
       \sloppy \large
       \sloppy

%\numberwithin{equation}{section}
%\newcommand{\sectio}[1]{
%\par
%\refstepcouner{section}
%{\normalsize\rmfamily\bfseries\upshape
%\thesection. #1.}
%}
\sloppy

\def\p{\partial}
\def\DS{\displaystyle}
\def\eps{\varepsilon}

\newcounter{FORMULA}
\renewcommand{\theFORMULA}{\arabic{FORMULA}}
       
       \def\Form#1$$#2$${\refstepcounter{FORMULA}%    %% здесь есть
%\\{\tt{#1}}                                  %% грязный трюк
$$#2\eqno(\theFORMULA)\label#1$$}              %% с label


\large
\renewcommand{\baselinestretch}{1.2}
\begin{document}

УДК 531.37

\smallskip


\centerline{2013~г.~{А.~Г.~Петров, С.~Е.~Володин}}

\smallskip

\centerline{\bf ОБ ЭФФЕКТЕ ДЖАНИБЕКОВА}
   
\bigskip

ВВЕДЕНИЕ

Космонавт Джанибеков наблюдал  движение гайки в космосе, кажущееся парадоксальным. Вращаясь вначале вокруг некоторой оси, гайка периодически совершала <<кувырок>>\,--- её ось вращения быстро поворачивалась на $ 180^\circ$. Этот эффект вытекает из закона сохранения кинетического момента, а процесс движения можно описать с помощью элементарных функций.

КИНЕМАТИЧЕСКИЕ И ДИНАМИЧЕСКИЕ ХАРАКТЕРИСТИКИ ТВЕРДОГО ТЕЛА

В опыте Джанибекова на твёрдое тело не действуют внешние силы. Поэтому его центр масс $C$ движется прямолинейно и равномерно. Поместим начало инерциальной системы отсчета в центр масс и будем изучать в ней движение твёрдого тела. Точка $C$ твёрдого тела является неподвижной  точкой в этой системн отсчета.

Для описания движения твёрдого тела около неподвижной точки $C$ кроме инерциальной системы отсчёта (ИСО) вводится подвижная, скреплённая с твёрдым телом система координат $\xi,\eta,\zeta$ с единичными базисными векторами ${\bf e}_1,{\bf e}_2,{\bf e}_3$. Ось $\eta$ и соответствующий ей базисный вектор ${\bf e}_2$ направим по оси гайки. Оси $\xi,\eta,\zeta$ являются главными осями тензора инерции твёрдого тела (рис. 1). 

Приведём общие формулы для  вектора угловой скорости ${\omega}$, кинетического момента ${\bf K}$ и кинетической энергии $T$
\Form{general}
$$
\begin{array}{l}
\omega=p{\bf e}_1+q{\bf e}_2+r{\bf e}_3,\quad {\bf K}=X{\bf e}_1+Y{\bf e}_2+Z{\bf e}_3,\quad
2T = X p+Y q+Z r
\end{array}
$$
где $X=Ap,\,Y=Bq,\,Z=Cr$ -- компоненты ${\bf K}$ и $A,B,C$ -- моменты инерции твёрдого тела, причем $A>B>C$.

В опыте Джанибекова интересно найти закон изменения угла $\theta$ между неподвижным вектором ${\bf K}$ и осью ${\bf e}_2$. Проектируя вектор ${\bf e}_2$ на вектор ${\bf K}$, получим
\Form{general1}
$$
\cos\theta=Y/K,\quad K=\sqrt{X^2+Y^2+Z^2}
$$
ЗАКОНЫ СОХРАНЕНИЯ 

При отсутствии внешних сил вектор кинетического момента ${\bf K}$ неподвижен в пространстве. Оси твёрдого тела ${\bf e}_1,{\bf e}_2,{\bf e}_3$  подвижны. Поэтому компоненты вектора ${\bf K}$ в подвижных осях 
 будут меняются со временем, однако квадат длины вектора сохраняется. Коме того, будет сохраняться кинетическая энергия твердого тела $2T=Ap^2+Bq^2+Cr^2=X^2/A+Y^2/B+Z^2/C$. Таким образом, траектория, которую описывает вектор ${\bf K}$ в подвижных осях, является пересечением сферы и эллипсоида
\Form{cons1}
$$
X^2+Y^2+Z^2=K^2,\quad  X^2/A+Y^2/B+Z^2/C=2T
$$
Такое представление траекторий называется геометрической интерпретацией Мак-Кулага \cite{jur}.
На рис. 2 изображено семейство траекторий вектора ${\bf K}$ на эллипсоиде Мак-Кулага. 

ДВИЖЕНИЕ ПО СЕПАРАТРИСАМ

В эксперименте Джанибекова
$A>B>C$. В начальный момент угловая скорость направлена по оси ${\bf e}_2$ и $Y=K,\;Y^2/B=2T$, следовательно выполняется соотношение $K^2=2TB$. Умножая второе уравнение на $B$ и вычитая из первого,
получим уравнение двух плоскостей
$$ 
(1-B/A)X^2-(B/C-1)Z^2=0\Rightarrow X\sqrt{1-B/A}\pm Z\sqrt{B/C-1}=0.
$$ 
Таким образом, при выполнении условия $K^2=2TB$ траектории вектора ${\bf K}$ являются пересечением плоскостей и эллипсоида Мак-Кулага. Это полуокружности между полюсами эллипсоида $B$ и $B'$. Они называются сепаратрисами и изображены на рис. 2. а)~--- на эллипсоиде Мак-Кулага, б)~--- развёртка сепаратрис.  

Из законов сохранения (\ref{cons1}) компоненты  вектора кинетического момента $X$ и $Z$ можно выразить через компоненту $Y$
\Form{cons2}
$$
X^2=\frac{B-C}{A-C}\,\frac{A}{B}(K^2-Y^2),\quad
Z^2=\frac{A-B}{A-C}\,\frac{C}{B}(K^2-Y^2)
$$
Уравнение для компоненты $Y$ можно получить, спроектировав закон сохранения кинетического момента в подвижных осях $d{\bf K}/dt+\omega\times K=0$ на ось ${\bf e}_2$ 
\Form{cons3}
$$
\frac{dY}{dt}=-(A-C)\,\frac{XZ}{AC}
$$   
Подставляя вместо $X$ и $Z$ выражения (\ref{cons2}), получим искомое уравнение
$$
\frac{dY}{dt}=\pm\frac{\alpha}{K}\left(K^2-Y^2\right), где\ \alpha =\sqrt{\frac{2T(A-B)(B-C)}{ABC}}=\frac{K}{B}\,\sqrt{\frac{(A-B)(B-C)}{AC}}\\
$$
Решение уравнения выражается через гиперболический тангенс. Отсюда и из (\ref{cons3}) находим компоненты вектора ${\bf K}$
$$
\begin{array}{c}
\DS X=\pm K\,\sqrt{\frac{(B-C)A}{B(A-C)}}\,\frac{1}{\ch\alpha t},\quad
\DS Z=\pm K\,\sqrt{\frac{(A-B)C}{B(A-C)}}\,\frac{1}{\ch\alpha t},\\
\DS Y=\pm K\th\alpha t
\end{array}
$$
Эти формулы известны и приведены, например, в монографии Аппеля \cite{appel}.

Согласно (\ref{general1}) угол $\theta$ между векторами ${\bf K}$ и ${\bf e}_2$ (угол нутации) меняется в интервале $(0, \pi)$ по закону
\Form{theta}
$$
\cos\theta=\pm\th\alpha t,
$$
поэтому движение вектора ${\bf K}$ от полюса $B'$ к полюсу $B$ происходит по одной из четырех полуокружностей за бесконечно большое время.

ИЗМЕНЕНИЕ УГЛА НУТАЦИИ ПРИ МАЛОМ ОТКЛОНЕНИИ ОТ СЕПАРАТРИСЫ.

Если имеется малое отклонение от сепаратрисы, которое можно охарактеризовать параметром
 $
 \eps=K^2-2TB,
 $
то  вектор ${\bf K}$ описывает замкнутую траекторию $MNN'M'$ близкую к сепаратрисам $BB'$  (рис.2) за конечное время. Половина этого времени, то есть, полупериод движения твердого тела $t_0$ находится следующим образом:

Из законов сохранения (\ref{cons1}) выражаем $X$ и $Z$ через $Y$  
$$
\begin{array}{c}
\DS X^2=\frac{B-C}{A-C}\,\frac{A}{B}(K^2-Y^2)+\frac{AC}{B(A-C)}\,\eps,\\[2ex]
\DS Z^2=\frac{A-B}{A-C}\,\frac{C}{B}(K^2-Y^2)-\frac{AC}{B(A-C)}\,\eps
\end{array}
$$
и подставляем их в уравнение (\ref{cons3}). Из него полупериод $t_0$ находится квадратурой
\Form{t0}
$$\DS t_0=2\frac{AC}{A-C}\int_0^{Y_0}\frac{dY}{X(Y)\,Z(Y)}$$

Верхний предел интеграла $Y_0$ является соответствующим знаку $\varepsilon$ корнем уравнений $X(Y)=0,\; Z(Y)=0$ (см. приложение).
 
Для вычисления полупериода при $\eps\rightarrow 0$ удобна асимптотика (см. приложение)
\Form{alphat0} 
$$
\DS\alpha t_0=\ln{\frac{16}{|\mu|}}+O(\mu\ln|\mu|), где\ \mu=\frac{\eps(A-C)B}{K^2(A-B)(B-C)}+o(\varepsilon)
$$ 
Зависимость $\cos\theta$ от времени можно скомбинировать с помощью функции (\ref{theta}), учитывая найденный полупериод $t_0$, т.к. вдали от полюсов $B$, $B'$ характер движения слабо зависит от $\eps$. На интервале $t\in (-t_0,5t_0)$ эта зависимость следующая (также см. рис. 3):
$$
\cos\theta=\th\alpha t - \th\alpha (t-t_0)+\th\alpha( t-2t_0)-\th\alpha (t-3t_0)+\th\alpha (t-4t_0)
$$

Наблюдатель в неподвижном пространстве увидит попеременное изменение направления средней оси твёрдого тела на противоположное. Поскольку скорость конца вектора $K$ вблизи полюсов близка к нулю, то средняя ось достаточно долго будет задерживаться вблизи полюса (дуга $MN$ на рис. 2)  и затем быстро изменит своё направление на противоположное (дуга $NN'$). При прохождении дуги $NN'$ совершается кувырок, при котором  $\cos\theta$ меняется от значения, близкого к 1, до значения, близкого к -1. В точках $N$ и $N'$ условимся считать, что 				$\cos\theta=0.9$ и $\cos\theta=-0.9$ соответственно. Время кувырка $\tau$ на дуге $NN'$, за которое $\cos\theta$ меняется от значения $0.9$ до  $-0.9$ найдётся из решения уравнения $\th(\alpha\tau/2)=0.9$. Это значение приблизительно равно $\tau=3/\alpha$.
Отношение $\tau$ к полупериоду движения при малом $\mu$ составляет малую величину
$$
\frac{\tau}{t_0}=3\left(\ln\frac{16}{|\mu|}\right)^{-1}
$$

Пусть в начальный момент твёрдое тело закручено около средней главной оси с малой погрешностью. Для компонент угловой скорости примем следующие начальные значения
$p_0,q_0,r_0$, $p_0^2+r_0^2<<q_0^2$. Тогда для модуля кинетического момента и параметров $\eps$ и $\mu$ имеем
$$
\begin{array}{c}
\DS K=Bq_{0},\quad \eps =A\left( A-B\right) p_{0}^{2}-C\left( B-C\right)
r_{0}^{2},\\[2ex]
\DS \mu = \frac{A-C}{C}\left(\frac{A}{B-C}\frac{p_0^2}{q_0^2}-\frac{C}{A-B}\frac{r_0^2}{q_0^2}\right)
\end{array}
$$
Для примера отношение времени кувырка к полупериоду равно $0.4$, откуда можно получить $\mu\approx 10^{-2}$. Считая, что $r_0=0$, можно получить $\frac{p_0}{q_0}\approx 0.05$, используя вычисленные моменты инерции (см. приложение).

\smallskip

ПРИЛОЖЕНИЕ

{\bf Расчёт главных моментов инерции.} Для приближенной оценки главных моментов инерции заменим изучаемую гайку цилиндром с внутренним и внешним радиусами цилиндрами $r_1,\,r_2$, высотой $h$ и массой $m_1$ и двумя симметрично расположенными относительно оси цилиндра точечными массами $m_2$ на расстоянии $a$ от оси цилиндра. Основание перпендикуляра от точечной массы $m_2$ на ось цилиндра находится на расстоянии $b$ от центра цилиндра. Массы $m_2$ крепятся к цилиндру стержнями, массы которых пренебрежимо малы. 

Центр масс такой системы $C$ находитcя на расстоянии $b_1$ от центра цилиндра $O$
$$
b_1=OC=\frac{2m_2}{2m_2+m_1}\,b,
$$
где масса $m_1=\rho\pi h(r_2^2-r_1^2).$
Центр системы координат $\xi,\eta,\zeta$ поместим в центр масс. Ось $\eta$ направлена по оси цилиндра, а ось $\zeta$ расположена в плоскости  оси цилиндра и точечных масс $m_2$. Определим моменты инерции относительно осей $\xi,\eta,\zeta$ цилиндра 
$$
A_1=C_1=\frac{m_1}{4}(r_1^2+r_2^2+\frac{h^2}{3}), B_1=\frac{m_1}{2}(r_1^2+r_2^2)
$$
двух масс $m_2$ 
$$
A_2=2m_2\left(a^2+(b-b_1)^2\right),\quad B_2=2m_2a^2,\quad C_2=2m_2(b-b_1)^2 
$$
и главные моменты инерции всей системы 
$$
A=A_1+A_2,\quad B=B_1+B_2,\quad C=C_1+C_2.
$$
Для должны выполняться соотношения $A>B>C$.

Выполнив вычисления, можно получить
$$A=2.12\times 10^{-7}кг м^2, B=1.74\times 10^{-7}кг м^2, C=0.62\times 10^{-7}кг м^2$$

{\bf Расчет периода.} Знак $\varepsilon$ определяет вид кривой, описываемой вектором $\vec{K}$: при положительных значениях эта кривая замкнута вокруг оси $\xi$, а при отрицательных~--- вокруг оси $\zeta$. Поэтому при $\varepsilon>0$ время достигает четверти периода при $Z(Y)=0$, а при $\varepsilon<0$~--- при $X(Y)=0$. Рассмотрим первый случай:
заменой переменной при $\eps>0$ 
$$
Y=x\sqrt{K^2-\frac{A}{A-B}\eps}
$$
интеграл для полупериода (\ref{t0})  приводится к виду
$$
\DS \alpha t_0=2\sqrt{1+\frac{B-C}{A-C}\mu}\int\limits_0^1\frac{dx}{\sqrt{(1-x^2)(1+\mu-x^2)}}
$$
где $\mu$~--- выражается через параметр $\eps$
\Form{mu}
$$\mu =\frac{\left( A-C\right) B\eps}{\left(
B-C\right) \left( \left( A-B\right) K^{2}-A\eps\right) },$$
а
$$\alpha=\sqrt{\frac{2T(A-B)(B-C)}{ABC}}.$$

Для интеграла можно получить асимптотическое разложение по малому параметру $\mu>0$
$$\begin{array}{l} \DS\int_{0}^{1}\frac{dy}{\sqrt{\left( 1-y^{2}\right) \left(
1+\mu -y^{2}\right) }}=\frac{1}{2}\left( \ln \frac{16}{\mu }\right) +\left(2+\ln 
\frac{\mu }{16}\right)\frac{\mu }{8}+\\[2ex]+\DS\frac{3}{128}\left(-7+3\ln \frac{16}{\mu }\right)\mu
^{2}+O(\mu ^{3}\ln \mu)\end{array} $$
С помощью него можно получить асимптотическое разложение для полупериода 
$$\alpha t_{0}=
\left( \ln \frac{16}{\mu }\right) \left[ 1+\left( \frac{1}{2\left( \ln \frac{%
16}{\mu }\right) }-\frac{A-2B+C}{4\left( A-C\right) }\right) \mu +O(\mu ^{2})%
\right] $$
В этом двухчленном разложение второе слагаемое в скобках определяет относительную ошибку, если ограничиться главным членом разложения.\\
Аналогично находится асимптотика (\ref{alphat0}) при $\eps<0$: заменой $Y=x\sqrt{K^2+\frac{C}{B-C}\varepsilon}$ интеграл (\ref{t0}) приводится к виду
$$
\alpha t_0=2\sqrt{1+\frac{A-B}{A-C}\mu}{\int\limits_{0}^{1}\frac{dx}{\sqrt{(1-x^2)(1+\mu-x^2)}}}, где
$$
$$
\mu=-\frac{B(A-C)\varepsilon}{\left(A-B\right)\left((B-C)K^2+C\varepsilon\right)}
$$
Отсюда получим главную асимптотику (\ref{alphat0}) при $\eps\rightarrow 0$, т.к. главные члены разложений совпадают.

\newpage


{ВЫВОДЫ} 


\smallskip
Таким образом, эффект, который наблюдал космонавт Джанибеков, описывается уравнениями Леонарда Эйлера, полученными и изученными им более 250 лет назад. Однако, в условиях невесомости этот эффект проявляется более ярко, на что и обратил внимание космонавт Джанибеков. 


\smallskip

Авторы благодарят В.Ф. Журавлева и Д.М. Климова за внимание к работе.

\begin{thebibliography}{99}
\bibitem{jur} Журавлев~В.~Ф. Основы теоретической механики.
М.:~Наука Физматлит, 1997, 320~с.

\bibitem{appel} Аппель П. Теоретическая механика. Том 2. М.: Физ-мат лит. 1960. 487 с.
\end{thebibliography}
\end{document}